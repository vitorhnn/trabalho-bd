\documentclass[12pt]{article}

\usepackage{sbc-template}
\usepackage{graphicx,url}
\usepackage{hyphenat}
\hyphenation{mate-mática recu-perar}
\usepackage[utf8]{inputenc}
\usepackage{listings}
\usepackage{xcolor}
\usepackage{amsmath}
\usepackage{amssymb}

\definecolor{backcolor}{rgb}{0.92,0.92,0.92}

\def\ojoin{\setbox0=\hbox{$\bowtie$}%
  \rule[-.02ex]{.25em}{.4pt}\llap{\rule[\ht0]{.25em}{.4pt}}}
\def\leftouterjoin{\mathbin{\ojoin\mkern-5.8mu\bowtie}}
\def\rightouterjoin{\mathbin{\bowtie\mkern-5.8mu\ojoin}}
\def\fullouterjoin{\mathbin{\ojoin\mkern-5.8mu\bowtie\mkern-5.8mu\ojoin}}

\renewcommand{\lstlistingname}{Código}
\lstdefinestyle{style}{
  backgroundcolor=\color{backcolor},
  captionpos=b,
  basicstyle=\ttfamily\scriptsize
}

\lstset{style=style}
     
\sloppy

\title{Relatório de Trabalho de Banco de Dados}

\author{Victor Hermann V. Chiletto}


\address{Departamento de Ciência da Computação \\Universidade Federal Rural do Rio de Janeiro (UFRRJ)
  \email{v@hnn.net.br}}

\begin{document} 

\maketitle

\section{Proposta do trabalho}

Foi proposto como trabalho a normalização até à terceira forma normal de um modelo lógico (código \ref{lst:modini}) e a realização de consultas em álgebra relacional e SQL no modelo lógico normalizado.

\begin{lstlisting}[caption=Modelo lógico inicial, label={lst:modini}]
Biblioteca (CodBib PK, NomeBiblioteca)

Obra(CodObra PK, TituloObra, AnoObra, EditoraObra, CodBib)
CodBib referencia Biblioteca

Emprestimo(CodObra PK, CodUsu PK, DataVencimento PK, NomeUsu)
  CodObra referencia Obra
\end{lstlisting}

\section{Normalizações}
  \subsection{Primeira forma normal}
    Como não há tabelas aninhadas, não foi necessário tomar nenhuma ação nesse passo.

  \subsection{Segunda forma normal}
    Há uma dependência funcional entre CodUsu e NomeUsu na relação Empréstimo. Criamos uma nova relação, Usuario, e removemos NomeUsu de Empréstimo para normalizar segundo a segunda forma normal.

    \begin{lstlisting}[caption=Modelo lógico na segunda forma normal, label={lst:mod2fn}]
Biblioteca (CodBib PK, NomeBiblioteca)

Obra(CodObra PK, TituloObra, AnoObra, EditoraObra, CodBib)
CodBib referencia Biblioteca

Usuario(CodUsu PK, NomeUsu)

Emprestimo(CodObra PK, CodUsu PK, DataVencimento PK)
CodObra referencia Obra
CodUsu referencia Usuario
    \end{lstlisting}
  
  \subsection{Terceira forma normal}
  Como não há dependências transitivas, não foi necesśario tomar nenhuma ação nesse passo.

\newpage

\section{Consultas em álgebra relacional}
Usaremos o modelo lógico descrito no código \ref{lst:mod2fn} para todas as consultas.

  \subsection{Excluir reservas que vencem em 20/12/2007 do usuário João Nascimento}
    \begin{align*}
\mathit{JNCod} &:= \Pi_{\mathit{CodUsu}}(\sigma(\mathit{NomeUsu} = \mathit{'João Nascimento'})(\mathit{Usuario})) \\
\mathit{Emprestimo} &\gets \mathit{Emprestimo} - (\sigma_{\mathit{DataVencimento} = \mathit{'2007-12-20'} \land \mathit{CodUsu} = \mathit{JNCod}}(\mathit{Emprestimo}))
    \end{align*}
  \subsection{Listar o nome do usuário e o título da obra dos empréstimos de cada usuário}
    \begin{align*}
\Pi_{\mathit{NomeUsu, CodObra}}(\mathit{Usuario} \bowtie \mathit{Emprestimo} \bowtie \mathit{Obra})
    \end{align*}
  \subsection{Exibir o nome da biblioteca e o número total de obras se o número total de obras for maior que 1000}
    \begin{align*}
\Pi_{\mathit{NomeBiblioteca}, \mathit{NumObras}}(\sigma_{\mathit{NumObras} > 1000}(\Gamma_{\operatorname{count}(\mathit{CodObra}) \operatorname{as} \mathit{NumObras}}(\mathit{Biblioteca} \leftouterjoin \mathit{Obra})))
    \end{align*}
  \subsection{Nomes dos usuários que não estão com empréstimos}
    \begin{align*}
      &\mathit{ComEmprestimo} := \Pi_{\mathit{CodUsu}, \mathit{NomeUsu}}(\mathit{Usuario} \bowtie \mathit{Emprestimo}) \\
      \Pi&_{\mathit{NomeUsu}}(\Pi_{\mathit{CodUsu}, \mathit{NomeUsu}}(\mathit{Usuario}) - \mathit{ComEmprestimo})
    \end{align*}
  \subsection{Três expressões equivalentes para selecionar o nome da biblioteca e título das obras da editora "ABC"}
    \begin{align*}
      \Pi&_{\mathit{NomeBiblioteca}, \mathit{TituloObra}}(\sigma_{\mathit{EditoraObra} = \mathit{'ABC'}}(\mathit{Obra} \bowtie \mathit{Biblioteca})) \\
      \Pi&_{\mathit{NomeBiblioteca}, \mathit{TituloObra}}(\sigma_{\mathit{Obra.CodBib} = \mathit{Biblioteca.CodBib} \land \mathit{Obra.EditoraObra} = \mathit{'ABC'}}(\mathit{Obra} \times \mathit{Biblioteca}))
    \end{align*}

\newpage

\section{Consultas em SQL}
Usamos os scripts no código \ref{lst:seedsql} para gerar o banco de dados para as seguintes consultas. Os scripts foram testados no sistema gerenciador de banco de dados PostgreSQL.\\\\
Anexado à este relatório há um \textit{Dockerfile} que usa o script do código \ref{lst:seedsql} para gerar uma imagem, o que pode ser usado como facilitador na hora de testar as consultas.

    \begin{lstlisting}[caption=Scripts em SQL para geração do banco de dados, label={lst:seedsql}, language=SQL]
CREATE TABLE Biblioteca(
    CodBib SERIAL PRIMARY KEY NOT NULL,
    NomeBiblioteca CHARACTER VARYING NOT NULL
);

CREATE TABLE Obra(
    CodObra SERIAL PRIMARY KEY NOT NULL,
    TituloObra CHARACTER VARYING NOT NULL,
    AnoObra INT NOT NULL,
    EditoraObra CHARACTER VARYING NOT NULL,
    CodBib INTEGER,

    FOREIGN KEY (CodBib) REFERENCES Biblioteca(CodBib)
);

CREATE TABLE Usuario(
    CodUsu SERIAL PRIMARY KEY NOT NULL,
    NomeUsu CHARACTER VARYING NOT NULL
);

CREATE TABLE Emprestimo(
    CodObra INTEGER NOT NULL,
    CodUsu INTEGER NOT NULL,
    DataVencimento DATE NOT NULL,

    PRIMARY KEY(CodObra, CodUsu, DataVencimento),
    FOREIGN KEY(CodObra) REFERENCES Obra(CodObra),
    FOREIGN KEY(CodUsu) REFERENCES Usuario(CodUsu)
);

INSERT INTO Biblioteca (NomeBiblioteca) VALUES
 ('Biblioteca do IM'),
 ('Biblioteca de Seropédica'),
 ('Biblioteca do CT');

INSERT INTO Usuario (NomeUsu) VALUES
  ('João Nascimento'),
  ('Victor Chiletto'),
  ('Maria Silva');

INSERT INTO Obra (TituloObra, AnoObra, EditoraObra, CodBib) VALUES 
  ('The C Programming Language, 2nd edition', 1988, 'Prentice Hall', 1),
  ('The C Programming Language, 2nd edition', 1988, 'Prentice Hall', 3),
  ('Um livro aleatório de 2004', 2004, 'Editora Aleatória', 2);

INSERT INTO Emprestimo (CodObra, CodUsu, DataVencimento) VALUES 
  (2, 1, '2007-12-20'),
  (1, 2, '2019-06-08'),
  (3, 2, '2019-12-06');
    \end{lstlisting}

  \newpage

    \subsection{Excluir todas as reservas que vencem em 20/12/2007 do usuário João Nascimento}
      \begin{lstlisting}[language=SQL]
DELETE FROM
  Emprestimo
WHERE
  DataVencimento = '2007-12-20'
  and CodUsu = (
    SELECT
      CodUsu
    FROM
      Usuario
    WHERE
      NomeUsu = 'João Nascimento'
  );
      \end{lstlisting}
    \subsection{Listar o nome do usuário e o título da obra dos empréstimos de cada usuário}
      \begin{lstlisting}[language=SQL]
SELECT
    NomeUsu, TituloObra
FROM
    Usuario
INNER JOIN
    Emprestimo ON Usuario.CodUsu = Emprestimo.CodUsu
INNER JOIN
    Obra ON Emprestimo.CodObra = Obra.CodObra;
      \end{lstlisting}
    \subsection{Exibir obras de 2004 que possuem reserva}
      \begin{lstlisting}[language=SQL]
SELECT DISTINCT
    TituloObra
FROM
    Obra
INNER JOIN
    Emprestimo ON Obra.CodObra = Emprestimo.CodObra
WHERE
    Obra.AnoObra = 2004;
      \end{lstlisting}
    \subsection{Exibir o nome da biblioteca e o número total de obras se o número total de obras for maior que 1000 obras}
      \begin{lstlisting}[language=SQL]
SELECT
    NomeBiblioteca, COUNT(Obra.CodObra) AS NumObras
FROM
    Biblioteca
LEFT OUTER JOIN
    Obra ON Biblioteca.CodBib = Obra.CodBib
GROUP BY
    NomeBiblioteca
HAVING
    COUNT(Obra.CodObra) > 1000;
      \end{lstlisting}
    \subsection{Nomes dos usuários que não estão com empréstimos}
      \begin{lstlisting}[language=SQL]
SELECT
    NomeUsu
FROM
    Usuario
WHERE
    Usuario.CodUsu NOT IN (
        SELECT
            InnerUsuario.CodUsu
        FROM
            Usuario as InnerUsuario
        INNER JOIN
            Emprestimo on InnerUsuario.CodUsu = Emprestimo.CodUsu
    );
      \end{lstlisting}

\end{document}
